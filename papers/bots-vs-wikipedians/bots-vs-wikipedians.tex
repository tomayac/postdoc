\documentclass{sig-alternate}

\usepackage[utf8]{inputenc}
\usepackage{enumitem}
\usepackage[hyphens]{url}
\usepackage[pdftex,urlcolor=black,colorlinks=true,linkcolor=black,citecolor=black]{hyperref}

\newcommand{\superscript}[1]{\ensuremath{^{\textrm{#1}}}}

% listings and Verbatim environment
\usepackage{fancyvrb}
\usepackage{relsize}
\usepackage{listings}
\usepackage{verbatim}
\newcommand{\defaultlistingsize}{\fontsize{8pt}{9.5pt}}
\newcommand{\inlinelistingsize}{\fontsize{8pt}{11pt}}
\newcommand{\smalllistingsize}{\fontsize{7.5pt}{9.5pt}}
\newcommand{\listingsize}{\defaultlistingsize}
\RecustomVerbatimCommand{\Verb}{Verb}{fontsize=\inlinelistingsize}
\RecustomVerbatimEnvironment{Verbatim}{Verbatim}{fontsize=\defaultlistingsize}
\lstset{frame=lines,captionpos=b,numberbychapter=false,escapechar=§,
        aboveskip=2em,belowskip=1em,abovecaptionskip=0.5em,belowcaptionskip=0.5em,
        framexbottommargin=-1em,basicstyle=\ttfamily\listingsize\selectfont}

% use Courier from this point onward
\let\oldttdefault\ttdefault
\renewcommand{\ttdefault}{pcr}
\let\oldurl\url
\renewcommand{\url}[1]{\inlinelistingsize\oldurl{#1}}

\lstdefinelanguage{JavaScript}{
  keywords={push, typeof, new, true, false, catch, function, return, null, catch, switch, var, if, in, while, do, else, case, break},
  keywordstyle=\bfseries,
  ndkeywords={class, export, boolean, throw, implements, import, this},
  ndkeywordstyle=\color{darkgray}\bfseries,
  identifierstyle=\color{black},
  sensitive=false,
  comment=[l]{//},
  morecomment=[s]{/*}{*/},
  commentstyle=\color{darkgray},
  stringstyle=\color{red},
  morestring=[b]',
  morestring=[b]"
}

% linewrap symbol
\usepackage{color}
\definecolor{grey}{RGB}{130,130,130}
\newcommand{\linewrap}{\raisebox{-.6ex}{\textcolor{grey}{$\hookleftarrow$}}}

% todo macro
\usepackage{color}
\newcommand{\todo}[1]{\noindent\textcolor{red}{{\bf \{TODO}: #1{\bf \}}}}

\begin{document}
%
% --- Author Metadata here ---
\conferenceinfo{International World Wide Web Conference}{2014 Seoul, Korea}
\CopyrightYear{2014} % Allows default copyright year (20XX) to be over-ridden - IF NEED BE.
%\crdata{0-12345-67-8/90/01}  % Allows default copyright data (0-89791-88-6/97/05) to be over-ridden - IF NEED BE.
% --- End of Author Metadata ---

\title{Bots vs. Wikipedians, Anons vs. Logged-In Humans:\\ A~Global Study of Edit Activity on Wikipedia and Wikidata}

\numberofauthors{1}

\author{
% 1st. author
\alignauthor
Thomas Steiner\titlenote{Thomas Steiner's second affiliation is \emph{Université de Lyon, CNRS Université Lyon~1, LIRIS, UMR5205, F-69622}}\\
       \affaddr{Google Germany GmbH}\\
       \affaddr{ABC-Str.~19}\\
       \affaddr{20354 Hamburg, Germany}\\
       \email{tomac@google.com}
}

\maketitle
\begin{abstract}
Wikipedia is a~global crowdsourced encyclopedia
that at time of writing is available in 287 languages.
Wikidata is a~likewise global crowdsourced knowledge base
that provides shared facts to be used by Wikipedias.
In the context of this research, we have developed
an application capable of monitoring
realtime edit activity of all language versions
of Wikipedia and Wikidata.
This application allows us and others to easily analyze edits
in order to answer questions such as
``Bots \emph{vs.}\ Wikipedians, who edits more?'',
``Anonymous \emph{vs.}\ logged-in humans, who edits what?'',
or ``Who are the bots and what do they edit?''.
To the best of our knowledge,
this is the first time such analyses
were done for really \emph{all} Wikipedias%
---big and small---and for Wikidata.
For evaluation purposes, our application is available publicly at
\url{http://wikipedia-edits.herokuapp.com/},
its code has been open-sourced under the Apache~2.0 license.
\end{abstract}

\category{ToDo}{\todo{category}}{}

\terms{\todo{terms}}

\keywords{\todo{keywords}}

\section{Introduction}

The fundamental shift from book-based encyclopedias
to CD-ROM-based encyclopedias
to finally Web-based encyclopedias
happened in the course of the 90ies
and started a~new era of freely and openly
available knowledge accessible to everybody.
The free online encyclopedia Wikipedia%
\footnote{\url{http://www.wikipedia.org/}}~\cite{sanger05historywikipedia} was formally launched
on January 15, 2001 by Jimmy Wales
and Larry Sanger,
albeit the fundamental technology and the underlying concepts are older.
Wikipedia's direct predecessor was Nupedia~\cite{sanger05historywikipedia},
a~similarly free online encyclopedia,
however, that was exclusively edited by experts
following a~strict peer-review process.
Wikipedia's initial role was to serve
for collaborating on draft articles for Nupedia.
What happened in practice was that Wikipedia rapidly overtook Nupedia
as there was no peer-review burden
and at time of writing is now a~global project
available in 287 languages and overall more than 30 million articles%
~\todo{stats reference}.
The international expansion began early on
in the project's existence,
with the first two non-English Wikipedias
being on March 16, 2001 the German and the Catalan ones,
followed briefly afterwards by (Romanized) Japanese.
What then followed was a~wave of new languages,
with French, Chinese, Dutch, Esperanto, Hebrew,
Italian, Portuguese, Russian, Spanish, Swedish,
Arabic, Hungarian, Afrikaans, Norwegian, and Serbian all being rolled out in the first year.

Wikidata\footnote{\url{http://www.wikidata.org/}}~\cite{vrandecic2012wikidata}
is a~free knowledge base that can be read
and edited by humans and machines alike.
The knowledge base centralizes access to
and management of structured data,
such as references between Wikipedias
and statistical information that can be used in articles.

\section{Related Work}

Wikipedias are ordered by hourly page views in recent days
Only Wikipedias which contain 10 or more articles and which received 10 or more edits in last month are listed above
Generated on Friday November 22, 2013 05:33 from recent database dump files.
Data processed up to Thursday October 31, 2013

Please note that the lengthy dump process (many weeks) means a~delay in publishing these statistics is always to be expected.


\url{http://stats.wikimedia.org/EN/TablesWikipediansContributors.htm}

\url{http://stats.wikimedia.org/EN/Sitemap.htm}

\url{http://en.wikipedia.org/wiki/Category:All_Wikipedia_bots}



\bibliographystyle{abbrv}
\bibliography{references}

\end{document}
