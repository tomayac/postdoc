\documentclass[runningheads,a4paper]{llncs}

\usepackage[utf8]{inputenc}

\usepackage{amssymb}
\setcounter{tocdepth}{3}
\usepackage{graphicx}

\newcommand{\keywords}[1]{\par\addvspace\baselineskip
\noindent\keywordname\enspace\ignorespaces#1}

\usepackage{pifont} 
\usepackage[utf8]{inputenc}
\usepackage{enumitem}
\usepackage[hyphens]{url}
\usepackage[pdftex,urlcolor=black,colorlinks=true,linkcolor=black,citecolor=black]{hyperref}
\def\sectionautorefname{Section}
\def\subsectionautorefname{Subsection}

\newcommand{\superscript}[1]{\ensuremath{^{\textrm{#1}}}}

% listings and Verbatim environment
\usepackage{fancyvrb}
\usepackage{relsize}
\usepackage{listings}
\usepackage{verbatim}
\newcommand{\defaultlistingsize}{\fontsize{8pt}{9.5pt}}
\newcommand{\inlinelistingsize}{\fontsize{8pt}{11pt}}
\newcommand{\smalllistingsize}{\fontsize{6.0pt}{7.0pt}}
\newcommand{\listingsize}{\smalllistingsize}%{\defaultlistingsize}
\RecustomVerbatimCommand{\Verb}{Verb}{fontsize=\inlinelistingsize}
\RecustomVerbatimEnvironment{Verbatim}{Verbatim}{fontsize=\defaultlistingsize}
\lstset{frame=lines,captionpos=b,numberbychapter=false,escapechar=§,
  aboveskip=2em,belowskip=1em,abovecaptionskip=0.5em,belowcaptionskip=0.5em,
  framexbottommargin=-1em,basicstyle=\ttfamily\listingsize\selectfont}

% use Courier from this point onward
\let\oldttdefault\ttdefault
\renewcommand{\ttdefault}{pcr}
\let\oldurl\url
%\renewcommand{\url}[1]{\defaultlistingsize\oldurl{#1}}

\usepackage[usenames,dvipsnames,svgnames,table]{xcolor}
\lstdefinelanguage{JavaScript}{
  keywords={push, typeof, new, true, false, catch, function, return, null,
    catch, switch, var, if, in, while, do, else, case, break, div, script, video},
  keywordstyle=\bfseries,
  ndkeywords={class, export, boolean, throw, implements, import, this},
  ndkeywordstyle=\color{darkgray}\bfseries,
  identifierstyle=\color{black},
  sensitive=false,
  comment=[l]{//},
  morecomment=[s]{/*}{*/},
  morecomment=[s]{<!--}{-->},  
  commentstyle=\color{darkgray},
  stringstyle=\color{green},
  morestring=[b]',
  morestring=[b]"
}
\lstset{breaklines=true}

% linewrap symbol
\usepackage{color}
\definecolor{grey}{RGB}{130,130,130}
\newcommand{\linewrap}{\raisebox{-.6ex}{\textcolor{grey}{$\hookleftarrow$}}}

% todo macro
\usepackage{color}
\newcommand{\todo}[1]{\noindent\textcolor{red}{{\bf \{TODO}: #1{\bf \}}}}

\def\JSONLD{\mbox{JSON-LD}}

\hyphenation{WebVTT}

\def\JSONLD{\mbox{JSON-LD}}

\begin{document}

\mainmatter  % start of an individual contribution

% first the title is needed
\title{Comprehensive Wikipedia Monitoring for Global and Realtime Natural Disaster Detection}

% a short form should be given in case it is too long for the running head
\titlerunning{Comprehensive Wikipedia Monitoring for Natural Disaster Detection}

% the name(s) of the author(s) follow(s) next
\author{
  Thomas Steiner
}
%
\authorrunning{Comprehensive Wikipedia Monitoring for Natural Disaster Detection}
% (feature abused for this document to repeat the title also on left hand pages)

% the affiliations are given next
\institute{
  Google Germany GmbH, Hamburg, Germany\ \ and\\
  CNRS, Université de Lyon, LIRIS -- UMR5205, Université Lyon~1, France\\  
  \email{tsteiner@\{liris.cnrs.fr, google.com\}}
}

\maketitle

\begin{abstract}
Natural disasters are harmful events
resulting from natural processes of the Earth.
Examples of natural disasters include tsunamis,
volcanic eruptions, earthquakes, floods, droughts,
and other geologic processes.
If they affect populated areas, natural disasters
can cause economic damage, injuries, or even losses of lives.
It is thus desirable that natural disasters
are detected as early as possible
and potentially affected persons be notified via emergency alerts.
By their pure nature, natural disasters are global phenomena
that people refer to by different names,
for example, the 2014 typhoon \emph{Rammasun}%
\footnote{Rammasun:
\url{https://en.wikipedia.org/wiki/Typhoon_Rammasun_(2014)}}
is known as typhoon \emph{Glenda} in the Philippines.
In this paper, we present our ongoing early-stage research
on a~realtime Wikipedia-based monitoring system
for the detection of natural disasters around the globe.
The long-term objective is to make data about natural disasters 
detected by this system available through public alerts
following the Common Alerting Protocol (CAP).

\keywords{Natural disaster detection, crisis response, Wikipedia}
\end{abstract}

\section{Introduction}

\subsection{Natural Disaster Detection and Response: a~Global Challenge}

According to a~study~\cite{laframboise2012naturaldisasters}
published in 2012 by the International Monetary Funds (IMF),
700~natural disasters were registered worldwide between 2010 and 2012,
affecting more than 450~million people.
According to the study, ``[d]amages have risen
from an estimated US\$20 billion on average per year
in the 1990s to about US\$100 billion per year during 2000--10.''
The authors expect this upward trend to continue
``as a~result of the rising concentration of people
living in areas more exposed to natural disasters,
and climate change.''
In consequence, public emergency alerting systems
become more and more crucial in the future.

National agencies like the
\emph{Federal Emergency Management Agency}
(FEMA)\footnote{FEMA: \url{http://www.fema.gov/}}
in the United States of America or the
\emph{Bundesamt für Bevölkerungsschutz und Katastrophenhilfe}
(BBK,\footnote{BBK: \url{http://www.bbk.bund.de/}}
``Federal Office of Civil Protection and Disaster Assistance'')
in Germany work to ensure the safety of the population
on a~national level, combining and providing relevant tasks
and information in a~single place.
The \emph{United Nations Office for the Coordination of Humanitarian Affairs}
(OCHA)\footnote{OCHA: \url{http://www.unocha.org/}}
is a~United Nations (UN) body formed to strengthen the UN's response
to complex emergencies and natural disasters.
The \emph{Global Disaster Alert and Coordination System}
(GDACS)\footnote{GDACS: \url{http://www.gdacs.org/}}
is ``a~cooperation framework between the United Nations,
the European Commission, and disaster managers worldwide
to improve alerts, information exchange, and coordination
in the first phase after major sudden-onset disasters.''
Global companies like Facebook,%
\footnote{Facebook Disaster Relief:
\url{https://www.facebook.com/DisasterRelief}}
Airbnb,\footnote{Airbnb Disaster Response:
\url{https://www.airbnb.com/disaster-response}} or
Google\footnote{Google Crisis Response:
\url{https://www.google.org/crisisresponse/}}
have dedicated crisis response teams that work on
making critical emergency information accessible in times of disaster.
As can be seen from the (incomprehensive) list above,
natural disaster detection and response is a~problem
tackled on national, international, and global levels;
both from the public and the private sectors.
To facilitate collaboration, a~common protocol is needed.

\subsection{The \emph{Common Alerting Protocol}}

The \emph{Common Alerting Protocol} (CAP)~\cite{westfall2010cap}
is an XML-based general data format for exchanging public warnings
and emergencies between alerting technologies.
CAP allows a~warning message to be consistently disseminated simultaneously
over many warning systems to many applications.
The protocol increases warning effectiveness and
simplifies the task of activating a~warning for officials.
CAP also provides the capability to include multimedia data,
such as photos, maps, or videos.
Alerts can be geographically targeted to a~defined warning area.
An exemplary flood warning CAP feed stemming from GDACS is shown in
\autoref{listing:cap}.

\begin{lstlisting}[caption={\emph{Common Alerting Protocol} feed
from the \emph{Global Disaster Alert and Coordination System} (\url{http://www.gdacs.org/xml/gdacs_cap.xml}, July~16, 2014)},
  label=listing:cap, language=xml,  morekeywords={xmlns,encoding,alert,identifier,sender,sent,status,msgType,scope,incidents,info,category,event,urgency,severity,certainty, senderName,headline,description,web,parameter,value,valueName,area,areaDesc,polygon},
  float=t!, stringstyle=\color{gray}, ]
<?xml version="1.0" encoding="utf-8"?>
<alert xmlns="urn:oasis:names:tc:emergency:cap:1.2">
  <identifier>GDACS_FL_4159_1</identifier>
  <sender>info@gdacs.org</sender>
  <sent>2014-07-14T23:59:59-00:00</sent>
  <status>Actual</status>
  <msgType>Alert</msgType>
  <scope>Public</scope>
  <incidents>4159</incidents>
  <info>
    <category>Geo</category>
    <event>Flood</event>
    <urgency>Past</urgency>
    <severity>Moderate</severity>
    <certainty>Unknown</certainty>
    <senderName>Global Disaster Alert and Coordination System</senderName>
    <headline />
    <description />
    <web>http://www.gdacs.org/reports.aspx?eventype=FL&amp;amp;eventid=4159</web>
    <parameter><valueName>eventid</valueName><value>4159</value></parameter>
    <parameter><valueName>currentepisodeid</valueName><value>1</value></parameter>
    <parameter><valueName>glide</valueName><value /></parameter>
    <parameter><valueName>version</valueName><value>1</value></parameter>
    <parameter><valueName>fromdate</valueName>
        <value>Wed, 21 May 2014 22:00:00 GMT</value></parameter>
    <parameter><valueName>todate</valueName>
        <value>Mon, 14 Jul 2014 21:59:59 GMT</value></parameter>
    <parameter><valueName>eventtype</valueName><value>FL</value></parameter>
    <parameter><valueName>alertlevel</valueName><value>Green</value></parameter>
    <parameter><valueName>alerttype</valueName><value>automatic</value></parameter>
    <parameter><valueName>link</valueName>
        <value>http://www.gdacs.org/report.aspx?eventtype=FL&amp;amp;eventid=4159</value>
    </parameter>
    <parameter><valueName>country</valueName><value>Brazil</value></parameter>
    <parameter><valueName>eventname</valueName><value /></parameter>
    <parameter><valueName>severity</valueName><value>Magnitude 7.44</value></parameter>
    <parameter><valueName>population</valueName><value>0 killed and 0 displaced</value>
    </parameter>
    <parameter><valueName>vulnerability</valueName><value /></parameter>
    <parameter><valueName>sourceid</valueName><value>DFO</value></parameter>
    <parameter><valueName>iso3</valueName><value /></parameter>
    <parameter><valueName>hazardcomponents</valueName>
        <value>FL,dead=0,displaced=0,main_cause=Heavy Rain,severity=2,sqkm=256564.57
        </value></parameter>
    <parameter><valueName>datemodified</valueName>
        <value>Mon, 01 Jan 0001 00:00:00 GMT</value></parameter>
    <area>
      <areaDesc>Polygon</areaDesc><polygon>,,100</polygon>
    </area>
  </info>
</alert>
\end{lstlisting}

\subsection{Contributions, Hypotheses, and Research Questions}

In this paper, we present first results of
our ongoing early-stage research
on a~realtime comprehensive Wikipedia-based monitoring system
for the detection of natural disasters around the globe.
We are steered by the following hypotheses.

\begin{itemize}
  \itemsep0em
  \item[($\mathbb{H}1$)] Content about natural disasters
    gets added to Wikipedia in a~timely fashion.
  \item[($\mathbb{H}2$)] Natural disasters being geographically
    constrained, textual and multimedia content about them
    gets added to local, \emph{i.e.}, non-English Wikipedias as well.
  \item[($\mathbb{H}3$)] Link structure dynamics of Wikipedia
    provide for a~meaningful way to detect future
    natural disasters, \emph{i.e.}, disasters unknown at system creation time.
\end{itemize}

\noindent These hypotheses lead us to the research questions below.

\begin{itemize}
  \itemsep0em
  \item[($\mathbb{Q}1$)] How timely and accurate for the purpose
    of natural disaster detection is content from Wikipedia
    compared to the official sources mentioned above?
  \item[($\mathbb{Q}2$)] Does the disambiguated nature of Wikipedia
    surpass keyword-based natural disaster detection approaches,
    \emph{e.g.}, via online social networks or search logs?
\end{itemize}

\section{Related Work}

Digitally crowdsourced data for disaster detection and response
has gained momentum in recent years,
as the Internet has proven resilient in times of crises,
compared to other infrastructure.
Ryan Falor, Crisis Response Product Manager at Google in 2011,
writes in~\cite{falor2011googleorg} that
``a~substantial \textup{[\,\dots]} proportion of searches
are directly related to the crises;
and people continue to search and access information online
even while traffic and search levels drop temporarily
during and immediately following the crises.''
In the following, we provide a~non-exhaustive list of related work
on digitally crowdsourced natural disaster detection and response.
Sakaki \emph{et~al.}\ consider in~\cite{sakaki2010earthquake} each user
of the online social network (OSN) site
Twitter\footnote{Twitter: \url{https://twitter.com/}} a~sensor
for the purpose of earthquake detection in Japan.
% They devise a~classifier of tweets based on features such as
% keywords, number of words, and their context;
% and produce a~probabilistic spatiotemporal model that
% can find the center and the trajectory of the earthquake.
Goodchild \emph{et~al.}\ show in~\cite{goodchild2010crowdsourcing}
how crowdsourced geodata from Wikipedia and
Wikimapia,\footnote{Wikimapia: \url{http://wikimapia.org/}}
``a~multilingual open-content collaborative map,''
can help complete authoritative data about natural disasters.
In~\cite{abel2012twitcident}, Abel \emph{et~al.}\ describe
a~crisis monitoring system that extracts relevant content
about known disasters from Twitter.
Liu \emph{et~al.}\ examine in~\cite{liu2008search}
common patterns and norms of natural disaster coverage
on the photo sharing platform Flickr.%
\footnote{Flickr: \url{https://www.flickr.com/}}
We have developed~\cite{steiner2014thesis} a~ monitoring system
that detects news events from concurrent Wikipedia edits
and auto-generates related multimedia galleries
based on content from various OSN sites
and Wikimedia Commons.\footnote{Wikimedia Commons: \url{https://commons.wikimedia.org/}}
Finally, Lin and Mishne examine realtime search query churn on Twitter%
~\cite{lin2012churn} including in the context of natural disasters.

\section{Proposed Methodology}

Wikipedia is an international online encyclopedia
currently available in 287~languages.%
\footnote{All Wikipedias: \url{https://meta.wikimedia.org/wiki/List_of_Wikipedias}}
\emph{(i)}~Articles in one language are interlinked with versions of the same article
in different languages, \emph{e.g.}, the article ``Natural disaster''
on the English Wikipedia
(\url{https://en.wikipedia.org/wiki/Natural_disaster}) \linebreak
links to 74~versions of the article in other languages.%
\footnote{Wikipedia language links:
\url{http://en.wikipedia.org/w/api.php?action=query&prop=langlinks&lllimit=max&titles=Natural_disaster}}
\emph{(ii)}~Each article can have redirects, \emph{i.e.}, different URLs
that point to the article.
For the English ``Natural disaster'' article, there are eight redirects,%
\footnote{Wikipedia redirects:
\url{http://en.wikipedia.org/w/api.php?action=query&list=backlinks&blfilterredir=redirects&bllimit=max&bltitle=Natural_disaster}}
\emph{e.g.}, ``Natural Hazard'' (synonym),
``Examples of natural disaster'' (refinement), or
``Natural disasters'' (plural).
\emph{(iii)}~For each article, the list of back links
that point to the article is available, \emph{i.e.},
inbound links other than redirects.
The article ``Natural disaster'' has more than 500 articles that link to it.%
\footnote{Wikipedia back links: \url{http://en.wikipedia.org/w/api.php?action=query&list=backlinks&bllimit=max&blnamespace=0&bltitle=Natural_disaster}}

Starting with the well-curated English article ``Natural disaster'',
we programmatically follow therein contained links of type
``See also:'' and ``Main article:''
that lead to an exhaustive list of English articles
of concrete instances of natural disasters,
\emph{e.g.}, ``Tsunami'' (\url{https://en.wikipedia.org/wiki/Tsunami}), ``Flood'' (\url{https://en.wikipedia.org/wiki/Flood}),
``Earthquake'' (\url{https://en.wikipedia.org/wiki/Earthquake}),
\emph{etc.} In total, we obtain links to 23~English articles
about different types of natural disasters.
For each of these English natural disasters articles,
we obtain all versions of each article in different languages
[step \emph{(i)} above],
and of the resulting list of international articles
in turn all their redirect URLs [step \emph{(ii)} above].
The end result is a~complete list of all articles in all Wikipedia languages
and all their redirects that are related to any type of natural disaster.
We make this list publicly available in different formats
(\texttt{.txt}, \texttt{.tsv}, \texttt{.json}), where the JSON version
is the most flexible and recommended one.%
\footnote{Natural disaster article list:
\url{https://github.com/tomayac/postdoc/tree/master/papers/comprehensive-wikipedia-monitoring-for-global-and-realtime-natural-disaster-detection/data}}


In the past, we have worked on a~Server-Sent Events API%
~\cite{steiner2014bots} capable of monitoring realtime edit activity
of all language versions of Wikipedia.
This API allows us to easily analyze Wikipedia edits

\bibliographystyle{abbrv}
\bibliography{references}
\end{document}
