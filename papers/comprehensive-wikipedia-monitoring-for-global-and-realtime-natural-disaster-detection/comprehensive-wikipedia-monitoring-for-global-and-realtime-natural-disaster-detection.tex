\documentclass[runningheads,a4paper]{llncs}

\usepackage[utf8]{inputenc}

\usepackage{amssymb}
\setcounter{tocdepth}{3}
\usepackage{graphicx}

\newcommand{\keywords}[1]{\par\addvspace\baselineskip
\noindent\keywordname\enspace\ignorespaces#1}

\usepackage{pifont} 
\usepackage[utf8]{inputenc}
\usepackage{enumitem}
\usepackage[hyphens]{url}
\usepackage[pdftex,urlcolor=black,colorlinks=true,linkcolor=black,citecolor=black]{hyperref}
\def\sectionautorefname{Section}
\def\subsectionautorefname{Subsection}

% listings and Verbatim environment
\usepackage{fancyvrb}
\usepackage{relsize}
\usepackage{listings}
\usepackage{verbatim}
\newcommand{\defaultlistingsize}{\fontsize{8pt}{9.5pt}}
\newcommand{\inlinelistingsize}{\fontsize{8pt}{11pt}}
\newcommand{\smalllistingsize}{\fontsize{6.5pt}{7.5pt}}
\newcommand{\listingsize}{\smalllistingsize}%{\defaultlistingsize}
\RecustomVerbatimCommand{\Verb}{Verb}{fontsize=\inlinelistingsize}
\RecustomVerbatimEnvironment{Verbatim}{Verbatim}{fontsize=\defaultlistingsize}
\lstset{frame=lines,captionpos=b,numberbychapter=false,escapechar=§,
        aboveskip=2em,belowskip=1em,abovecaptionskip=0.5em,belowcaptionskip=0.5em,
        framexbottommargin=-1em,basicstyle=\ttfamily\listingsize\selectfont}

% use Courier from this point onward
\let\oldttdefault\ttdefault
\renewcommand{\ttdefault}{pcr}
\let\oldurl\url
\renewcommand{\url}[1]{\inlinelistingsize\oldurl{#1}}

\usepackage[usenames,dvipsnames,svgnames,table]{xcolor}
\lstdefinelanguage{JavaScript}{
  keywords={push, typeof, new, true, false, catch, function, return, null,
    catch, switch, var, if, in, while, do, else, case, break, div, script, video},
  keywordstyle=\bfseries,
  ndkeywords={class, export, boolean, throw, implements, import, this},
  ndkeywordstyle=\color{darkgray}\bfseries,
  identifierstyle=\color{black},
  sensitive=false,
  comment=[l]{//},
  morecomment=[s]{/*}{*/},
  morecomment=[s]{<!--}{-->},  
  commentstyle=\color{darkgray},
  stringstyle=\color{green},
  morestring=[b]',
  morestring=[b]"
}
\lstset{breaklines=true}

% linewrap symbol
\usepackage{color}
\definecolor{grey}{RGB}{130,130,130}
\newcommand{\linewrap}{\raisebox{-.6ex}{\textcolor{grey}{$\hookleftarrow$}}}

% todo macro
\usepackage{color}
\newcommand{\todo}[1]{\noindent\textcolor{red}{{\bf \{TODO}: #1{\bf \}}}}

\def\JSONLD{\mbox{JSON-LD}}

\hyphenation{WebVTT}

\def\JSONLD{\mbox{JSON-LD}}

\begin{document}

\mainmatter  % start of an individual contribution

% first the title is needed
\title{Comprehensive Wikipedia Monitoring for Global and Realtime Natural Disaster Detection}

% a short form should be given in case it is too long for the running head
\titlerunning{Comprehensive Wikipedia Monitoring for Natural Disaster Detection}

% the name(s) of the author(s) follow(s) next
\author{
  Thomas Steiner
}
%
\authorrunning{Comprehensive Wikipedia Monitoring for Natural Disaster Detection}
% (feature abused for this document to repeat the title also on left hand pages)

% the affiliations are given next
\institute{
  Google Germany GmbH, Hamburg, Germany\ \ and\\
  CNRS, Université de Lyon, LIRIS -- UMR5205, Université Lyon~1, France\\  
  \email{tsteiner@\{liris.cnrs.fr, google.com\}}
}

\maketitle

\begin{abstract}
Natural disasters are harmful events
resulting from natural processes of the Earth.
Examples of natural disasters include tsunamis,
volcanic eruptions, earthquakes, floods, droughts,
and other geologic processes.
If they affect populated areas, natural disasters
can cause economic damage, injuries, or even losses of lives.
It is thus desirable that natural disasters
are detected as early as possible
and potentially affected persons be notified via emergency alerts.
By their pure nature, natural disasters are global phenomena
that people refer to by different names,
for example, the 2014 typhoon \emph{Rammasun}%
\footnote{Rammasun:
\url{https://en.wikipedia.org/wiki/Typhoon_Rammasun_(2014)}}
is known as typhoon \emph{Glenda} in the Philippines.
In this paper, we present our ongoing early-stage research
on a~realtime Wikipedia-based monitoring system
for the detection of natural disasters around the globe.
The long-term objective is to make data about natural disasters 
detected by this system available through public alerts
following the Common Alerting Protocol (CAP).

\keywords{Natural disaster detection, crisis response, Wikipedia}
\end{abstract}

\section{Introduction}

\subsection{Natural Disaster Detection and Response: a~Global Challenge}

According to a~study~\cite{laframboise2012naturaldisasters}
published in 2012 by the International Monetary Funds (IMF),
700~natural disasters were registered worldwide between 2010 and 2012,
affecting more than 450~million people.
According to the study, ``[d]amages have risen
from an estimated US\$20 billion on average per year
in the 1990s to about US\$100 billion per year during 2000--10.''
The authors expect this upward trend to continue
``as a~result of the rising concentration of people
living in areas more exposed to natural disasters,
and climate change.''
In consequence, public emergency alerting systems
become more and more crucial in the future.

National agencies like the
\emph{Federal Emergency Management Agency}
(FEMA)\footnote{FEMA: \url{http://www.fema.gov/}}
in the United States of America or the
\emph{Bundesamt für Bevölkerungsschutz und Katastrophenhilfe}
(BBK,\footnote{BBK: \url{http://www.bbk.bund.de/}}
``Federal Office of Civil Protection and Disaster Assistance'')
in Germany work to ensure the safety of the population
on a~national level, combining and providing relevant tasks
and information in a~single place.
The \emph{United Nations Office for the Coordination of Humanitarian Affairs}
(OCHA)\footnote{OCHA: \url{http://www.unocha.org/}}
is a~United Nations (UN) body formed to strengthen the UN's response
to complex emergencies and natural disasters.
The \emph{Global Disaster Alert and Coordination System}
(GDACS)\footnote{GDACS: \url{http://www.gdacs.org/}}
is ``a~cooperation framework between the United Nations,
the European Commission, and disaster managers worldwide
to improve alerts, information exchange, and coordination
in the first phase after major sudden-onset disasters.''
Global companies like Facebook,%
\footnote{Facebook Disaster Relief:
\url{https://www.facebook.com/DisasterRelief}}
Airbnb,\footnote{Airbnb Disaster Response:
\url{https://www.airbnb.com/disaster-response}} or
Google\footnote{Google Crisis Response:
\url{https://www.google.org/crisisresponse/}}
have dedicated crisis response teams that work on
making critical emergency information accessible in times of disaster.
As can be seen from the (incomprehensive) list above,
natural disaster detection and response is a~problem
tackled on national, international, and global levels;
both from the public and the private sectors.
To facilitate collaboration, a~common protocol is needed.
An exemplary CAP feed stemming from GDACS can be seen in
\autoref{listing:cap}.

\subsection{The \emph{Common Alerting Protocol}}

The \emph{Common Alerting Protocol} (CAP)~\cite{westfall2010cap}
is an XML-based general data format for exchanging public warnings
and emergencies between alerting technologies.
CAP allows a~warning message to be consistently disseminated simultaneously
over many warning systems to many applications.
The protocol increases warning effectiveness and
simplifies the task of activating a~warning for officials.
CAP also provides the capability to include multimedia data,
such as photos, maps, or videos.
Alerts can be geographically targeted to a~defined warning area. 

\begin{lstlisting}[caption={\emph{Common Alerting Protocol} feed
from the \emph{Global Disaster Alert and Coordination System} (\url{http://www.gdacs.org/xml/gdacs_cap.xml}, July~16, 2014)},
  label=listing:cap, language=xml,
  morekeywords={xmlns,encoding,alert,identifier,sender,sent,status,msgType,scope,incidents,info,category,event,urgency,severity,certainty, senderName,headline,description,web,parameter,value,valueName,area,areaDesc,polygon},
  float=t!, stringstyle=\color{gray}, ]
<?xml version="1.0" encoding="utf-8"?>
<alert xmlns="urn:oasis:names:tc:emergency:cap:1.2">
  <identifier>GDACS_FL_4159_1</identifier>
  <sender>info@gdacs.org</sender>
  <sent>2014-07-14T23:59:59-00:00</sent>
  <status>Actual</status>
  <msgType>Alert</msgType>
  <scope>Public</scope>
  <incidents>4159</incidents>
  <info>
    <category>Geo</category>
    <event>Flood</event>
    <urgency>Past</urgency>
    <severity>Moderate</severity>
    <certainty>Unknown</certainty>
    <senderName>Global Disaster Alert and Coordination System</senderName>
    <headline />
    <description />
    <web>http://www.gdacs.org/reports.aspx?eventype=FL&amp;amp;eventid=4159</web>
    <parameter><valueName>eventid</valueName><value>4159</value></parameter>
    <parameter><valueName>currentepisodeid</valueName><value>1</value></parameter>
    <parameter><valueName>glide</valueName><value /></parameter>
    <parameter><valueName>version</valueName><value>1</value></parameter>
    <parameter><valueName>fromdate</valueName>
        <value>Wed, 21 May 2014 22:00:00 GMT</value></parameter>
    <parameter><valueName>todate</valueName>
        <value>Mon, 14 Jul 2014 21:59:59 GMT</value></parameter>
    <parameter><valueName>eventtype</valueName><value>FL</value></parameter>
    <parameter><valueName>alertlevel</valueName><value>Green</value></parameter>
    <parameter><valueName>alerttype</valueName><value>automatic</value></parameter>
    <parameter><valueName>link</valueName>
        <value>http://www.gdacs.org/report.aspx?eventtype=FL&amp;amp;eventid=4159</value>
    </parameter>
    <parameter><valueName>country</valueName><value>Brazil</value></parameter>
    <parameter><valueName>eventname</valueName><value /></parameter>
    <parameter><valueName>severity</valueName><value>Magnitude 7.44</value></parameter>
    <parameter><valueName>population</valueName><value>0 killed and 0 displaced</value>
    </parameter>
    <parameter><valueName>vulnerability</valueName><value /></parameter>
    <parameter><valueName>sourceid</valueName><value>DFO</value></parameter>
    <parameter><valueName>iso3</valueName><value /></parameter>
    <parameter><valueName>hazardcomponents</valueName>
        <value>FL,dead=0,displaced=0,main_cause=Heavy Rain,severity=2,sqkm=256564.57
        </value></parameter>
    <parameter><valueName>datemodified</valueName>
        <value>Mon, 01 Jan 0001 00:00:00 GMT</value></parameter>
    <area>
      <areaDesc>Polygon</areaDesc><polygon>,,100</polygon>
    </area>
  </info>
</alert>
\end{lstlisting}

\subsection{Contributions, Hypotheses, and Research Questions}

In this paper, we present first results of
our ongoing early-stage research
on a~realtime comprehensive Wikipedia-based monitoring system
for the detection of natural disasters around the globe.
We are steered by the following hypotheses.

\begin{itemize}
  \itemsep0em
  \item[($\mathbb{H}1$)] Textual and multimedia content about
    natural disasters gets added to Wikipedia in a~timely fashion.
  \item[($\mathbb{H}2$)] Natural disasters being geographically
    constrained, content about them gets added to local,
    \emph{i.e.}, non-English Wikipedias as well.
  \item[($\mathbb{H}3$)] The link structure of Wikipedia provides
    for a~meaningful way to discover natural disasters of the future.
\end{itemize}

\noindent These hypotheses lead us to the research questions below.

\begin{itemize}
  \itemsep0em
  \item[($\mathbb{Q}1$)] What breaking news event features
    determine the relevancy of the corresponding media gallery?
  \item[($\mathbb{Q}2$)] What factors determine the choice
    of the preferred media gallery kind for a~breaking news event?
\end{itemize}



\section{Related Work}


\bibliographystyle{abbrv}
\bibliography{references}
\end{document}
