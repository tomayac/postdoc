\documentclass{sig-alternate}

\usepackage[utf8]{inputenc}
\usepackage{enumitem}
\usepackage[hyphens]{url}
\usepackage[pdftex,urlcolor=black,colorlinks=true,linkcolor=black,citecolor=black]{hyperref}
\def\sectionautorefname{Section}
\def\subsectionautorefname{Subsection}

\usepackage{CJKutf8}

\newcommand{\superscript}[1]{\ensuremath{^{\textrm{#1}}}}

% listings and Verbatim environment
\usepackage[usenames,dvipsnames,svgnames,table]{xcolor}
\usepackage{fancyvrb}
\usepackage{relsize}
\usepackage{listings}
\usepackage{verbatim}
\newcommand{\defaultlistingsize}{\fontsize{8pt}{9.5pt}}
\newcommand{\inlinelistingsize}{\fontsize{8pt}{11pt}}
\newcommand{\smalllistingsize}{\fontsize{7.5pt}{9.5pt}}
\newcommand{\listingsize}{\defaultlistingsize}
\RecustomVerbatimCommand{\Verb}{Verb}{fontsize=\inlinelistingsize}
\RecustomVerbatimEnvironment{Verbatim}{Verbatim}{fontsize=\defaultlistingsize}
\lstset{frame=lines,captionpos=b,numberbychapter=false,escapechar=§,
        aboveskip=2em,belowskip=1em,abovecaptionskip=0.5em,belowcaptionskip=0.5em,
        framexbottommargin=-1em,basicstyle=\ttfamily\listingsize\selectfont}

% use Courier from this point onward
\let\oldttdefault\ttdefault
\renewcommand{\ttdefault}{pcr}
\let\oldurl\url
\renewcommand{\url}[1]{\inlinelistingsize\oldurl{#1}}

\lstdefinelanguage{JavaScript}{
  keywords={console, log, addEventListener, onmessage, alert, push, typeof, new, true, false, catch, function, return, null, catch, switch, var, if, in, while, do, else, case, break},
  keywordstyle=\bfseries,
  ndkeywords={class, export, boolean, throw, implements, import, this},
  ndkeywordstyle=\color{darkgray}\bfseries,
  identifierstyle=\color{Maroon},
  sensitive=false,
  comment=[l]{//},
  morecomment=[s]{/*}{*/},
  commentstyle=\color{ForestGreen},
  stringstyle=\color{Blue},
  morestring=[b]',
  morestring=[b]"
}

% linewrap symbol
\usepackage{color}
\definecolor{grey}{RGB}{130,130,130}
\newcommand{\linewrap}{\raisebox{-.6ex}{\textcolor{grey}{$\hookleftarrow$}}}

% todo macro
\usepackage{color}
\newcommand{\todo}[1]{\noindent\textcolor{red}{{\bf \{TODO}: #1{\bf \}}}}

\begin{document}
%
% --- Author Metadata here ---
\conferenceinfo{International Conference on Multimedia Retrieval}{2014 Glasgow, UK}
\CopyrightYear{2014} % Allows default copyright year (20XX) to be over-ridden - IF NEED BE.
%\crdata{0-12345-67-8/90/01}  % Allows default copyright data (0-89791-88-6/97/05) to be over-ridden - IF NEED BE.
% --- End of Author Metadata ---

\title{Telling Breaking News Stories from Wikipedia with Social Multimedia: A~Case Study of the 2014 Winter Olympics}

\numberofauthors{1}

\author{
% 1st. author
\alignauthor
Jon Doe\\%Thomas Steiner\titlenote{Thomas Steiner's second affiliation is \emph{Université de Lyon, CNRS Université Lyon~1, LIRIS, UMR5205, F-69622}}\\
  \affaddr{Random University}\\%\affaddr{Google Germany GmbH}\\\affaddr{Google Germany GmbH}\\
  \affaddr{Random Street 0, 12345 Random City}\\       %\affaddr{ABC-Str.~19, 20354 Hamburg, Germany}\\
       \email{jon.doe@nospam.com}%\email{tomac@google.com}
}

\maketitle
\begin{abstract}
With the ability to watch Wikipedia
and Wikidata edits in realtime,
the online encyclopedia and the knowledge base
have become increasingly used targets of research
for the detection of breaking news events.
In this paper, we present a~case study of the
\emph{2014 Winter Olympics}, where we tell the story of
breaking news events in the context of the Olympics
with the help of social multimedia
stemming from multiple social networks.
Therefore, we have extended the application
\emph{Wikipedia Live Monitor}---%
a~tool for the detection of breaking news events---%
with the capability of automatically creating
media galleries that illustrate events.
Athletes winning an Olympic competition,
a~new country leading the medal table,
or simply the Olympics themselves are all events
newsworthy enough for people to concurrently
edit Wikipedia and Wikidata---%
around the world in many languages.
The Olympics being an event of common interest,
an even bigger majority of people share the event
in a~multitude of languages on global social networks.
This sharing of moments in the form of comments and multimedia
happens either in people's role as spectators or participants
directly at one of the Olympic sites,
or as indirect second screen users
in front of their TV~sets at home.
With this work, we connect the world of
Wikipedia and Wikidata with the world of social networks,
in order to convey the spirit of the
\emph{2014 Winter Olympics},
to tell the story of victory and defeat,
following the Olympic motto \emph{Citius, Altius, Fortius}.

\end{abstract}

\category{H.5.1}{Information Interfaces and Presentation}{Multimedia Information Systems}

\terms{Human Factors, Languages, Measurement, Experimentation}

\keywords{Storytelling, social networks, multimedia, Wikipedia}

\section{Introduction}

\bibliographystyle{abbrv}
\bibliography{references}
\balancecolumns
\end{document}