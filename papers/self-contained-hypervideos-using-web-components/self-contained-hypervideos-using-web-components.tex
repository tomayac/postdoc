\documentclass[runningheads,a4paper]{llncs}

\usepackage[utf8]{inputenc}

\usepackage{amssymb}
\setcounter{tocdepth}{3}
\usepackage{graphicx}

\newcommand{\keywords}[1]{\par\addvspace\baselineskip
\noindent\keywordname\enspace\ignorespaces#1}

\usepackage{pifont} 
\usepackage[utf8]{inputenc}
\usepackage{enumitem}
\usepackage[hyphens]{url}
\usepackage[pdftex,urlcolor=black,colorlinks=true,linkcolor=black,citecolor=black]{hyperref}
\def\sectionautorefname{Section}
\def\subsectionautorefname{Subsection}

% listings and Verbatim environment
\usepackage{fancyvrb}
\usepackage{relsize}
\usepackage{listings}
\usepackage{verbatim}
\newcommand{\defaultlistingsize}{\fontsize{8pt}{9.5pt}}
\newcommand{\inlinelistingsize}{\fontsize{8pt}{11pt}}
\newcommand{\smalllistingsize}{\fontsize{6.0pt}{7.0pt}}
\newcommand{\listingsize}{\smalllistingsize}%{\defaultlistingsize}
\RecustomVerbatimCommand{\Verb}{Verb}{fontsize=\inlinelistingsize}
\RecustomVerbatimEnvironment{Verbatim}{Verbatim}{fontsize=\defaultlistingsize}
\lstset{frame=lines,captionpos=b,numberbychapter=false,escapechar=§,
  aboveskip=2em,belowskip=1em,abovecaptionskip=0.5em,belowcaptionskip=0.5em,
  framexbottommargin=-1em,basicstyle=\ttfamily\listingsize\selectfont}

% use Courier from this point onward
\let\oldttdefault\ttdefault
\renewcommand{\ttdefault}{pcr}
\let\oldurl\url
%\renewcommand{\url}[1]{\defaultlistingsize\oldurl{#1}}

\usepackage[usenames,dvipsnames,svgnames,table]{xcolor}
\lstdefinelanguage{JavaScript}{
  keywords={push, typeof, new, true, false, catch, function, return, null,
    catch, switch, var, if, in, while, do, else, case, break, div, script, video},
  keywordstyle=\bfseries,
  ndkeywords={class, export, boolean, throw, implements, import, this},
  ndkeywordstyle=\color{darkgray}\bfseries,
  identifierstyle=\color{black},
  sensitive=false,
  comment=[l]{//},
  morecomment=[s]{/*}{*/},
  morecomment=[s]{<!--}{-->},  
  commentstyle=\color{darkgray},
  stringstyle=\color{green},
  morestring=[b]',
  morestring=[b]"
}
\lstset{breaklines=true}

% linewrap symbol
\usepackage{color}
\definecolor{grey}{RGB}{130,130,130}
\newcommand{\linewrap}{\raisebox{-.6ex}{\textcolor{grey}{$\hookleftarrow$}}}

% todo macro
\usepackage{color}
\newcommand{\todo}[1]{\noindent\textcolor{red}{{\bf \{TODO}: #1{\bf \}}}}

\def\JSONLD{\mbox{JSON-LD}}

\hyphenation{WebVTT}

\def\JSONLD{\mbox{JSON-LD}}

\begin{document}

\mainmatter  % start of an individual contribution

% first the title is needed
\title{Self-Contained Hypervideos\\ Using Web Components}

% a short form should be given in case it is too long for the running head
\titlerunning{Self-Contained Hypervideos Using Web Components}

% the name(s) of the author(s) follow(s) next
\author{
  Thomas Steiner\textsuperscript{1} \and
  Pierre-Antoine Champin\textsuperscript{1} \and \\
  Benoît Encelle\textsuperscript{1}\and
  Yannick Prié\textsuperscript{2}
}
%
\authorrunning{Steiner, Champin, Encelle, and Prié}
% (feature abused for this document to repeat the title also on left hand pages)

% the affiliations are given next
\institute{
  \textsuperscript{1}CNRS, Université de Lyon, LIRIS -- UMR5205, Université Lyon~1, France\\
  \email{\{tsteiner, pierre-antoine.champin\}@liris.cnrs.fr, benoit.encelle@univ-lyon1.fr}\\
  \textsuperscript{2}CNRS, Université de Nantes, LINA -- UMR 6241, France\\
  \email{yannick.prie@univ-nantes.fr}
}

\maketitle

\begin{abstract}
The creation of hypervideos---displayed video streams
that contain embedded user-clickable anchors and annotations---%
is a~manual and tedious job,
requiring the preparation of assets like still frames,
the segmentation of videos in scenes or chapters,
and sometimes even the isolation of objects
like faces within the video.
In this paper, we propose a~semi-automated
Web-Components-based approach to self-contained hypervideo creation.
By \emph{self-contained} we mean that all necessary intrinsic components
of the hypervideo, \emph{e.g.}, still frames,
should come from the video itself
rather than be included as external assets.
\emph{Web Components} is a~set of specifications,
which let Web developers apply their HTML, CSS,
and JavaScript knowledge to build widgets
that can be reused easily and reliably.
By leveraging this evolving standard,
we obtain a~high degree of abstraction,
which reduces the burden of creating hypervideos
to the familiar task of textually marking them up with HTML elements.

\keywords{Hypervideo, Web Components, semantic video annotation}
\end{abstract}

\section{Introduction}

The term \emph{hypervideo} is commonly used to refer to
\textit{``a~displayed video stream that contains embedded user-clickable anchors''}%
~\cite{sawhney1996hypercafe,smith2002extensible}
and annotations, allowing for navigation between the video and other hypermedia elements.
In a~2006 article in \emph{The Economist}, the authors write 
\textit{``[h]yperlinking video involves the use of ``object-tracking'' software
to make filmed objects, such as cars, clickable as they move around.
Viewers can then click on items of interest in a~video
to watch a related clip; after it has played,
the original video resumes where it left off.
To inform viewers that a~video is hyperlinked,
editors can add highlights to moving images, use beeps as audible cues,
or display still images from hyperlinked videos
next to the clip that is currently playing''}~\cite{economist2006hypervideo}.
In standard literature, hypervideo is considered a~logical consequence
of the related concept of \emph{hypertext}.
In contrast to hypertext, hypervideo necessarily includes a~time component,
as content changes over time.
In consequence, hypervideo has different technical and aesthetic requirements
than hypertext, the most obvious one being appropriate segmentation in scenes
or even objects.

\section{Related Work}

Related work can be regarded under the angles
of online annotation creation and large-scale Linked Data 
efforts for video.
Many have combined Linked Data and video,
typical examples are~\cite{lambert2010linkeddata} by Lambert \emph{et~al.}\
and~\cite{hausenblas2009im} by Hausenblas \emph{et~al.}
There are several text track enriching approaches~\cite{li2013enriching,li2012creating,yi2012synote,steiner2010semwebvid},
all centered around the application of named entity recognition.
The online video hosting platform YouTube
lets video publishers add video annotations
in a~closed proprietary format.
From 2009 to 2010, YouTube had a~feature called
Collaborative Annotations%
~\cite{fink2009collaborativeannotations}
that allowed video consumers to collaboratively
create video annotations.
In~\cite{vandeursen2012mediafragmentannotations},
Van Deursen \emph{et~al.}\ present a~system
that combines Media Fragments URI~\cite{troncy2012mediafragments}
and the Ontology for Media Resources~\cite{lee2012mediaontology}
in an HTML5 Web application to convert
rich media fragment annotations into a~WebVTT file
that can be used by HTML5-enabled players
to show the annotations in a~synchronized way.
Building on their work, in~\cite{steiner2014webvtt},
we additionally allowed for writing annotations by
letting annotators create WebVTT cues with an editor.
Popcorn.js\footnote{Popcorn.js: \url{http://popcornjs.org/}}
is an HTML5 JavaScript media framework
for the creation of media mixes
by adding interactivity and context to videos
by letting users link social media, feeds,
visualizations, and other content to moving images.
PopcornMaker%
\footnote{PopcornMaker: \url{https://popcorn.webmaker.org/}}
is an interactive Web authoring environment
allowing for videos to be annotated on a~timeline.
Popcorn video annotations are essentially JavaScript~programs.

\bibliographystyle{abbrv}
\bibliography{references}
\end{document}
